%%%%%%%%%%%%%%%%%%%%%%%%%%%%%%%%%%%%%%%%%
% Journal Article
% LaTeX Template
% Version 1.1 (25/11/12)
%
% This template has been downloaded from:
% http://www.LaTeXTemplates.com
%
% Original author:
% Frits Wenneker (http://www.howtotex.com)
%
% License:
% CC BY-NC-SA 3.0 (http://creativecommons.org/licenses/by-nc-sa/3.0/)
%
%%%%%%%%%%%%%%%%%%%%%%%%%%%%%%%%%%%%%%%%%

%----------------------------------------------------------------------------------------
%	PACKAGES AND OTHER DOCUMENT CONFIGURATIONS
%----------------------------------------------------------------------------------------

\documentclass[twoside]{article}

\usepackage{lipsum} % Package to generate dummy text throughout this template

\usepackage[sc]{mathpazo} % Use the Palatino font
\usepackage[T1]{fontenc} % Use 8-bit encoding that has 256 glyphs
\linespread{1.05} % Line spacing - Palatino needs more space between lines
\usepackage{microtype} % Slightly tweak font spacing for aesthetics

\usepackage[hmarginratio=1:1,top=32mm,columnsep=20pt]{geometry} % Document margins
\usepackage{multicol} % Used for the two-column layout of the document
\usepackage{hyperref} % For hyperlinks in the PDF

\usepackage[hang, small,labelfont=bf,up,textfont=it,up]{caption} % Custom captions under/above floats in tables or figures
\usepackage{booktabs} % Horizontal rules in tables
\usepackage{float} % Required for tables and figures in the multi-column environment - they need to be placed in specific locations with the [H] (e.g. \begin{table}[H])

\usepackage{lettrine} % The lettrine is the first enlarged letter at the beginning of the text
\usepackage{paralist} % Used for the compactitem environment which makes bullet points with less space between them

\usepackage{abstract} % Allows abstract customization
\renewcommand{\abstractnamefont}{\normalfont\bfseries} % Set the "Abstract" text to bold
\renewcommand{\abstracttextfont}{\normalfont\small\itshape} % Set the abstract itself to small italic text

\usepackage{titlesec} % Allows customization of titles
\renewcommand\thesection{\Roman{section}}
\titleformat{\section}[block]{\large\scshape\centering}{\thesection.}{1em}{} % Change the look of the section titles

\usepackage{fancyhdr} % Headers and footers
\pagestyle{fancy} % All pages have headers and footers
\fancyhead{} % Blank out the default header
\fancyfoot{} % Blank out the default footer
\fancyhead[C]{Running title $\bullet$ November 2012 $\bullet$ Vol. XXI, No. 1} % Custom header text
\fancyfoot[RO,LE]{\thepage} % Custom footer text

%----------------------------------------------------------------------------------------
%	TITLE SECTION
%----------------------------------------------------------------------------------------

\title{\vspace{-15mm}\fontsize{24pt}{10pt}\selectfont\textbf{Comment Abuse Analyzer}} % Article title

\author{
\large
\textsc{Clinton Pahl, James Osgood, Paul Vandermeer}\thanks{A thank you or further information}\\[2mm] % Your name
\normalsize University of Alberta \\ % Your institution
\normalsize \href{mailto:cpahl@ualberta.ca}{cpahl@ualberta.ca} % Your email address
\normalsize \href{mailto:osgood@ualberta.ca}{osgood@ualberta.ca} % Your email address
\normalsize \href{mailto:pv1@ualberta.ca}{pv1@ualberta.ca} % Your email address
\vspace{-5mm}
}
\date{}

%----------------------------------------------------------------------------------------

\begin{document}

\maketitle % Insert title

\thispagestyle{fancy} % All pages have headers and footers

%----------------------------------------------------------------------------------------
%	ABSTRACT
%----------------------------------------------------------------------------------------

\begin{abstract}

\noindent
Since the world is depending more and more to computers to feed people information there tends \
there is a chance that there will be more and more abuse posted.  Not everyone wants to see these \
messages as some of them could be profound and very abusive to some.  Our project is going to predict \
whether there is abuse detected in a publicly post message.  To solve this problem we have tried a couple \
of different methods.


\end{abstract}

%----------------------------------------------------------------------------------------
%	ARTICLE CONTENTS
%----------------------------------------------------------------------------------------

\begin{multicols}{2} % Two-column layout throughout the main article text

\section{Problem}
Our project entails detecting abuse in public online comments.  This is a common problem that is getting worse \
each and everyday.  For website administrators or moderators to manually go and moderate all the comments on their \
website is very costly and time consuming.  Since peforming this task is time consuming a lot of stuff does get missed \
or does not ever get looked at.  If every comment was to be looked at first posts were take a long time to actually hit \
website and people would be turned off and go elsewhere.  The big problem we are faced with is whether certain comments \
are considered abuse or if they are just regular comments.  Since profanity can be used in a regular unabusive way the \
training data used needs to be refinded so that different comments are classified differently.

%------------------------------------------------

\section{Methods}

The methods we are going to investigate are: Bootstrapping, Bag of Words, String Kernel.  We chose these \
methods as we hypothesized that it would be easier to parse out the features we require.  Here is just a summary \
of what these methods entail.\
\\
\\
\textbf{Bootstrapping}\\
Lets make a dataset D$_{n}$ where n is the number of samples in the datatset.  \
Then we take j = 1 $\rightarrow$ k, where k is how many sampples we are going to use for bootstrapping.  \
This new dataset will be chosen by random where the j$_{th}$ entry will replace the current entry \
in D$_{n}$.  After this process is done we will end up with a modified dataset D$_{n}$ that has some \
entries in duplicate and it may also might have removed some items.
\\
\\
\textbf{Bag of words}\\
With the bag of words algorithm we take all the words of the training set and uniquely place them in a \
new ordered set.  The words in the the phrases now create a matrix where each row is create by a method \
that introduces a weight count for each time the word exists.  This special method called "Term weighting" uses \
hashing to efficiently compare which words in the comment are in our dictionary and thus limits the resources \
required to run our algorithm.  \

 

%------------------------------------------------

\section{Literature survey}

%------------------------------------------------

\section{Approach}

%------------------------------------------------

\section{Rational}

%------------------------------------------------

\section{Hypotheses}

%------------------------------------------------

\section{Experimental design}

%------------------------------------------------

\section{Results}

%------------------------------------------------

\section{Critical evaluation}

%------------------------------------------------

\section{Lessons learned}

%----------------------------------------------------------------------------------------
%	REFERENCE LIST
%----------------------------------------------------------------------------------------

\begin{thebibliography}{99} % Bibliography - this is intentionally simple in this template

\bibitem[Figueredo and Wolf, 2009]{Figueredo:2009dg}
Figueredo, A.~J. and Wolf, P. S.~A. (2009).
\newblock Assortative pairing and life history strategy - a cross-cultural
  study.
\newblock {\em Human Nature}, 20:317--330.
 
\end{thebibliography}

%----------------------------------------------------------------------------------------

\end{multicols}

\end{document}
